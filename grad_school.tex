\documentstyle[12pt]{article}
\setlength{\oddsidemargin}{0in}
\setlength{\evensidemargin}{0in}
\setlength{\textwidth}{6.5in}
\setlength{\topmargin}{-.65in}
\setlength{\textheight}{9in}
\pagestyle{empty}

\begin{document}

\begin{center}
    {\Large Statement of Purpose} \\[3 mm]
    {\large Ryan L. Drapeau}
\end{center}

\vspace*{.075in}

I live for the theoretical, the hypothetical, the what-if. The endless possibilities of theorizing, the profound knowledge of simulated experiments that just begs for further analysis is what personally draws me to computer science. My interest started the moment {\it{Jeopardy!}} champion Ken Jennings lost to IBM's supercomputer, Watson, back in 2011; an unfathomably complicated combination of deep learning and Bayesian inference created a system capable of defeating a master at his own game. I became fascinated with the goal of one day working on a project that would have similar impact.\vspace{1 mm}

In my undergraduate studies, I have had the honor of researching with two distinguished faculty members. As I entered my second quarter, I started to work in machine learning with Dr.\ Carlos Guestrin to model user uncertainty in online data collection. Shortly thereafter I joined the mobile accessibility group with Dr.\ Richard Ladner to develop a braille typewriter for iOS called Perkinput. By working with both professors concurrently, I received a depth of exposure in their respective fields as well as a breadth of knowledge that spans the core areas of computer science. Exploring these projects with an inquisitive attitude led to a continuous learning experience that brought me well beyond the confines of the classroom.\vspace{1 mm}

This year I began work with Johan Ugander (a PhD candidate under Dr.\ Guestrin) to understand how variability in people’s uncertainty affects ``wisdom of the crowd'' mechanisms in online data collection. Over the course of the project, I learned about Amazon’s Mechanical Turk and how the system could be used to collect incredible amounts of information from workers. With Johan’s guidance, I launched fifteen different experiments collecting data from thousands of users. My favorite part came in the analysis phase of this research where I wrote many different Python programs to filter out bad users, perform bootstrap sampling in parallel, and evaluate estimator performance. I find it difficult to replicate the exhilaration experienced from commanding clusters of computers while in the comfort of my dorm room. Pending countless {\LaTeX} and analysis adjustments, this project will be my first publication and submission to NIPS.\vspace{1 mm}

Research allows me to learn about topics that are never taught in the classroom; it enables me to explore the unknown, through application of the theoretical. While designing Perkinput’s nonvisual interface, I had to find a way to see what my users could not; I was forced to shift my own perspective so that I could develop a better product. The first version of Perkinput was incredibly difficult to use for someone who was blind. However, after several more iterations, I learned how to increase accessibility while maintaining the usability of eyes-free text entry. Generous use of Apple’s VoiceOver, a screen reader that is used by the visually impaired, ensures that the user knows and can interact with the state of the program at all times. With Perkinput’s improved interface, the app was featured on {\it{AppleVis}}, a podcast that targets over 150,000 visually impaired listeners every month, and continues to improve with every week. I received the Mary Gates Research Scholarship for my work on Perkinput and I plan to continue refining the app using feedback from users as a guide. I will be directly applying the skills I gained through this project at Google Research this summer, where I will be working with Dr.\ T.V. Raman to make Google Glass more accessible.\vspace{1 mm}

As a future graduate student, I cannot imagine life without challenge – I thrive from it. Whether I am designing an accessible application or developing a new model for data collection, I strive to continuously learn and contribute my knowledge to the academic community. It is this process that continuously inspires and drives innovation in computer science. [Closing Sentence].

\end{document}
