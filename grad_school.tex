\documentstyle[12pt]{article}
\setlength{\oddsidemargin}{0in}
\setlength{\evensidemargin}{0in}
\setlength{\textwidth}{6.5in}
\setlength{\topmargin}{-.95in}
\setlength{\textheight}{9.75in}
\pagestyle{empty}

\begin{document}

\begin{center}
    {\Large Statement of Purpose}\\[2 mm]
    {\large Ryan Lee Drapeau}
\end{center}

My interest in computer science started the moment {\it{Jeopardy!}}\ champion Ken Jennings lost to IBM's supercomputer, Watson, in 2011; I remember being awed as the unfathomable machine learning system defeated a human master at his own game. Through my studies, I have learned how Watson is an elegant synthesis of deep learning and Bayesian inference atop a complex distributed computing platform. I am fascinated with the goal of working on projects at the forefront of computer science that can have similar impact.\vspace{2 mm}

As an undergrad, I explored different aspects of computer science research under two distinguished faculty members. During my second quarter of college, I employed theories of machine learning with Prof.\ Carlos Guestrin to model user uncertainty in online data collection. Shortly thereafter I joined the mobile accessibility group with Prof.\ Richard Ladner to develop a braille keyboard for iOS called Perkinput. By concurrently working with both professors, I received a depth of exposure in their respective fields as well as a breadth of knowledge that spans the core areas of computer science. Research allows me to learn about topics that are never taught in the classroom; it enables me to explore questions that don't necessarily have answers, to take what I've learned and derive exciting new consequences.\vspace{2 mm}

Last September I began my senior thesis work with Johan Ugander, a Ph.D.\ candidate under Prof.\ Guestrin, to understand how an individual's confidence affects ``wisdom of the crowd'' mechanisms. With Johan's guidance, I launched fifteen online experiments, collecting data from thousands of unique international users. In the analysis phase of the research, I used empirical bootstrap sampling, parallelized for efficiency, to evaluate various statistical estimators and their performance. Our work shows that estimator accuracy is improved when users' uncertainty is incorporated into the model. Johan and I are currently in the final stages of the analysis for this project, and have begun preparing a paper to be submitted during the early fall.\vspace{2 mm}

With Prof.\ Ladner, I worked extensively on a consumer-driven project for blind iOS users. While designing Perkinput's nonvisual interface, I had to find a way to see what my users could not. I was forced to shift my own perspective so that I could develop a better product. I learned fundamental concepts of accessible design, which require a balance between considerations for the sighted and the visually impaired, and then applied these concepts over several interface iterations. With Perkinput's improved interface, the app was featured on AppleVis, a podcast that targets over 150,000 listeners every month. I will be directly applying the skills I gained through this project at Google Research this summer, where I will be working with Dr.\ T.V.\ Raman to make Google Glass more accessible.\vspace{2 mm}

As a future graduate student, I cannot imagine life without challenge – I thrive from it. Whether I am designing an accessible application or developing a statistical model for data collection, I strive to continuously learn and contribute my knowledge to both the academic community and the industry at large. It is this process that constantly inspires and drives innovation in this fast growing field. I am on track to graduate a year early, yet I want to continue exploring machine learning and mobile accessibility through my current research, as well as through graduate-level classes. The combined B.S./M.S.\ program will allow me to achieve the depth of knowledge required by the difficult research problems I plan to fill my career with. This degree will act as an important stepping-stone in my career, whether my path keeps me in industry or leads me back to academia for a Ph.D.\
\end{document}
