\documentstyle[12pt]{article}
\setlength{\oddsidemargin}{0in}
\setlength{\evensidemargin}{0in}
\setlength{\textwidth}{6.5in}
\setlength{\topmargin}{-.95in}
\setlength{\textheight}{9.75in}
\pagestyle{empty}

\begin{document}

\begin{center}
    {\Large Statement of Purpose}\\[1 mm]
    {\large Ryan L. Drapeau}
\end{center}

I live for the theoretical, the hypothetical, the what if. The endless possibilities of theorizing, the profound knowledge emerging from simulated experiments that just begs for further analysis is what personally draws me to computer science. My interest started the moment Jeopardy! champion Ken Jennings lost to IBM’s supercomputer, Watson, back in 2011; an unfathomably complicated combination of deep learning and Bayesian inference created a system capable of defeating a master at his own game. I became fascinated with the goal of one day working on a project that would have similar impact.\vspace{2 mm}

As an undergrad, I explored different aspects of computer science research under two distinguished faculty members. During my second quarter of college, I employed theories of machine learning with Prof. Carlos Guestrin to model user uncertainty in online data collection. Shortly thereafter I joined the mobile accessibility group with Prof. Richard Ladner to develop a braille keyboard for iOS called Perkinput. By concurrently working with both professors, I received a depth of exposure in their respective fields as well as a breadth of knowledge that spans the core areas of computer science. Research allows me to learn about topics that are never taught in the classroom; it enables me to explore the unknown, through application of the theoretical.\vspace{2 mm}

This school year I began my senior thesis work with Johan Ugander (a Ph.D. candidate under Prof. Guestrin) to understand how an individual’s confidence affects “wisdom of the crowd” mechanisms. With Johan’s guidance, I launched fifteen different experiments, collecting data from thousands of unique international users. In the analysis phase of the research, I used empirical bootstrap sampling in parallel to evaluate various estimators (definition of estimator, if room) and their performance. Our work shows that estimator accuracy is improved when users’ uncertainty is incorporated into the model. Pending countless LaTeX and analysis adjustments, this project will be my first publication and submission to the Neural Information Processing Systems conference (NIPS).\vspace{2 mm}

With Prof. Ladner, I worked extensively on a consumer-driven project with consumers I could not easily connect with. While designing Perkinput’s nonvisual interface, I had to find a way to see what my users could not. I was forced to shift my own perspective so that I could develop a better product. I learned fundamental concepts of accessible design, which require a balance between considerations for the sighted and the visually impaired, and then applied these concepts over several interface iterations. With Perkinput’s improved interface, the app was featured on AppleVis, a podcast that targets over 150,000 visually impaired listeners every month. I received the Mary Gates Research Scholarship for my work on Perkinput and I plan to continue refining the app by integrating feedback from users. I will be directly applying the skills I gained through this project at Google Research this summer, where I will be working with Dr. T.V. Raman to make Google Glass more accessible.\vspace{2 mm}

As a future graduate student, I cannot imagine life without challenge – I thrive from it. Whether I am designing an accessible application or developing a statistical model for data collection, I strive to continuously learn and contribute my knowledge to both the academic community and the industry at large. It is this process that constantly inspires and drives innovation in this fast growing field. The combined B.S./M.S. program will allow me to further benefit from well-respected faculty to deepen my knowledge of computer science before I pursue a competitive job in my chosen field. This degree will act as an important stepping-stone in my career, whether my path keeps me in industry or leads me back to academia for a Ph.D.
\end{document}
