\documentstyle[12pt]{article}
\setlength{\oddsidemargin}{0in}
\setlength{\evensidemargin}{0in}
\setlength{\textwidth}{6.5in}
\setlength{\topmargin}{-.65in}
\setlength{\textheight}{9in}
\pagestyle{empty}

\begin{document}

\begin{center}
    {\Large Statement of Purpose} \\[3 mm]
    {\large Ryan L. Drapeau}
\end{center}

\vspace*{.075in}

I live for the theoretical, the hypothetical, the what-if. The endless possibilities of theorizing, the profound knowledge of simulated experiments that just begs for further analysis is what personally draws me to computer science. My interest started the moment Jeopardy! champion Ken Jennings lost to IBM’s supercomputer, Watson, back in 2011; an unfathomably complicated combination of deep learning and Bayesian inference created a system capable of defeating a master at his own game. I became fascinated with the goal of one day working on a project that would have similar impact.\vspace{3 mm}

In my undergraduate studies, I have had the honor of researching with two distinguished faculty members. As I entered my second quarter at the University of Washington, I started to work in machine learning with Dr. Carlos Guestrin. Shortly thereafter I joined the mobile accessibility group with Dr. Richard Ladner. By working with both professors concurrently, I received a depth of exposure in their respective fields as well as a breadth of knowledge that spans the core areas of computer science. Exploring these projects with an inquisitive attitude led to a continuous learning experience that brought me well beyond the confines of the classroom.\vspace{3 mm}

Machine Learning Here\vspace{3 mm}

Research allows me to learn about topics that are never taught in the classroom; it enables me to explore the unknown, through application of the theoretical. While designing the iOS application for the blind, I had to find a way to see what my users could not. I was forced to shift my own perspective so that I could develop a better product. The first version of Perkinput was incredibly difficult to use if the screen was not visible. However, after several more iterations, I learned how to increase accessibility while maintaining the usability of eyes-free text entry. Generous use of Apple’s VoiceOver, a screen reader that is used by the visually impaired, ensures that the user knows the state of the program at all times.\vspace{3 mm}

I received the Mary Gates Research Scholarship for my work on Perkinput and I plan to continue refining the application using feedback from users as a guide. I will be directly applying the skills I gained through this project at Google Research this summer, where I will be working with Dr. T.V. Raman to make Google Glass more accessible.\vspace{3 mm}

I cannot imagine life without challenge – I thrive from it. Whether I am designing an application for the blind or developing a new model for data collection, I strive to continuously learn and contribute my knowledge to the academic community. It is this process that continuously inspires and drives innovation in computer science.\vspace{3 mm}

\end{document}