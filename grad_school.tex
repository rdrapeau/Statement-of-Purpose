\documentstyle[12pt]{article}
\setlength{\oddsidemargin}{0in}
\setlength{\evensidemargin}{0in}
\setlength{\textwidth}{6.5in}
\setlength{\topmargin}{-.65in}
\setlength{\textheight}{9.5in}
\pagestyle{empty}

\begin{document}

\begin{center}
    {\Large Statement of Purpose}\\[3 mm]
    {\large Ryan L. Drapeau}
\end{center}

I live for the theoretical, the hypothetical, the what-if. The endless possibilities of theorizing,
the profound knowledge of simulated experiments that just begs for further analysis is what personally draws me to computer science. My interest started the moment {\it{Jeopardy!}} champion Ken Jennings lost to IBM's supercomputer, Watson, back in 2011; an unfathomably complicated combination of deep learning and Bayesian inference created a system capable of defeating a master at his own game. I became fascinated with the goal of one day working on a project that would have similar impact.\vspace{2 mm}

As an undergrad, I had the honor of researching with two distinguished faculty members. During my second quarter, I employed theories of machine learning with Prof.\ Carlos Guestrin to model user uncertainty in online data collection. Shortly thereafter I joined the mobile accessibility group with Prof.\ Richard Ladner to develop a braille typewriter for iOS called Perkinput. By working with both professors concurrently, I received a depth of exposure in their respective fields as well as a breadth of knowledge that spans the core areas of computer science. Exploring these projects with an inquisitive attitude led to a continuous learning experience that brought me well beyond the confines of the classroom.\vspace{2 mm}

This year I began work with Johan Ugander (a PhD candidate under Prof.\ Guestrin) to understand how an individual's confidence affects ``wisdom of the crowd'' mechanisms. Machine learning often requires incredible amounts of data collected from services like Amazon's Mechanical Turk; our work shows that incorporating a user's uncertainty yields better estimators. With Johan's guidance, I launched fifteen different experiments, collecting data from thousands of unique international users. In the analysis phase of the research, I performed bootstrap sampling in parallel to evaluate various estimators and their accuracy. Pending countless \LaTeX{} and analysis adjustments, this project will be my first publication and submission to NIPS.\vspace{2 mm}

Research allows me to learn about topics that are never taught in the classroom; it enables me to explore the unknown, through application of the theoretical. While designing Perkinput's nonvisual interface, I had to find a way to see what my users could not; I was forced to shift my own perspective so that I could develop a better product. After several iterations, I learned how to increase accessibility while maintaining the usability of eyes-free text entry. With Perkinput's improved interface, the app was featured on {\it{AppleVis}}, a podcast that targets over 150,000 visually impaired listeners every month. I received the Mary Gates Research Scholarship for my work on Perkinput and I plan to continue refining the app, integrating feedback from users. I will be directly applying the skills I gained through this project at Google Research this summer, where I will be working with Dr.\ T.V.\ Raman to make Google Glass more accessible.\vspace{2 mm}

As a future graduate student, I cannot imagine life without challenge – I thrive from it. Whether I am designing an accessible application or developing a statistical model for data collection, I strive to continuously learn and contribute my knowledge to both the academic community and the industry at large. It is this process that continuously inspires and drives innovation in computer science, and while I do not plan on pursuing a PhD immediately, after a few years in industry I do plan on returning to academic research.
\end{document}
